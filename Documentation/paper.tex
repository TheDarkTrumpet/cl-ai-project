\documentclass{article}

\usepackage[english]{babel}
\usepackage[pdftex]{graphicx}
\usepackage[pdftex,bookmarks=true,breaklinks=true,bookmarksnumbered=true]{hyperref}
\usepackage{listings}

%Use package listings here for the syntax highlighting.

\hypersetup{
  pdfauthor = {David Thole}
  pdftitle = {22C145 - Artificial Intelligence Class Project}
  pdfsubject = {22C145 - Artificial Intelligence}
  pdfkeywords = {Artificial Intelligence,22C145,AI}
}

\begin{document}

\title{22C:145 - Class Project}
\author{David Thole}
\date{07May2010}

% Table of Contents:
\pdfbookmark[1]{Contents}{toc}  %%% additional bookmark for ToC
\thispagestyle{plain}           %%% uses the above defined fancy page-header
\tableofcontents
\markboth{Table of Contents}{Table of Contents} %%% for the page header
\cleardoublepage                %%% start again on odd page

\section{Introduction}
\subsection{Data Sets Picked, language choice}
The data set I picked for this programming problem is the \textbf{mushroom data set}, which is located http://archive.ics.uci.edu/ml/datasets/Mushroom \\
The reason I picked this data set is because it's a binary classification problem with lots of examples.  I also wanted to test with a data set that was incredibly large, and while there are larger data sets available - this one had the right distribution through the class and feature attributes that I felt it was an ``easy'' data set to use.  I wanted to approach this problem more about trying to build efficient ways of computing - rather than a data set that is super interesting. \\
\\
The language I picked for this problem is \textbf{Common Lisp}, the reason why I picked this is mostly due to my comfort level in the language, and that there is little in the mining area when it comes to this particular language.  I've been toying with creating applications for work that would relate to text mining, and figured some of the concepts of what I would do for this class would hopefully work out here too.

\subsection{Installation and Running}
This tarball contains all the files needed to run this application, minus 1.  This applicaton relies on a library called fare-csv.  To install that, you will need to run:

\begin{lstlisting}[frame=single,language=lisp]
sbcl --dynamic-space-size
(require :asdf-install)
(asdf-install:install 'fare-csv)
\end{lstlisting}

To run the individual examples, there are a few differnet functions available:
\begin{lstlisting}[frame=single,language=lisp]
(run-everything)
(run-nb [folds])
(run-nn [folds])
\end{lstlisting}

An example, of running the naive bayes algorithm on the default data, with 10-fold cross validation, would be:
\begin{lstlisting}[frame=single,language=lisp]
(run-nb 10)
\end{lstlisting}

An important thing to note, that you'll likely run into is that it is looking for the data file (verify this is a problem before handing in)

To set a differnet file to be analyzed, all you need to do is run the following:
\begin{lstlisting}[frame=single,language=lisp]
(setf *data-set-file* #P"path/to/file.csv")
\end{lstlisting}

\subsection{Design Choices - Algorithms and Code Structure}
The code is extremely well documented in my  opinion, so this section will be fairly short.  I built the code in such a way that it was more plugin based - with some areas heavily copied to make it easier for the reader to follow.  Ideally, there would be less copy and paste in this project, but it works the way it is now.  The graphic on the following pages explains an a picture way how the code is organized.  Each algorithm is represented as a self-encapsulated module of sorts - you can load each package without reliance on any other - except from the ai package.  I did this so the project could grow without needing to worry about name clashing.  With that in mind, most of the code that'll be of immediate interest is in the lib/main.lisp file.  This file is the starting point of the for what is run, and the analysis done.  This file is split up roughly into 2 chunks - one being dedicated to running the naive bayes algorithm and analysis of it, and the other for the k-nearest neighbors and that analysis.  I comment in the file with lots of ``;'' to denote what is going where.  Much of the stuff is duplicated for easy reading.  The next file of interest would be the lib/utils.lisp, which contains how I load the data file, get attributes/class values and so on.  In the individual packages, main.lisp will be the main file of interest for each algorithm.\\
\\
The algorithm I picked, besides the requested naive bayes, was k-nearest neighbors which was already mentioned.  I ran into a number of issues related to performance with this algorithm, but I'll go over that a bit later.

\rotatebox{90}{
  \includegraphics[scale=0.75]{flowchart.png}
}

\section{Algorithms and Performance Measures}
\subsection{Tabular Accuracy Measures}
For the values below, I worked at trying 20-fold cross validation to try and get the most accurate average possible.  On the apparent error rate, I went with far less of a fold amount so that it actually finishes (since it's testing over far more data).  Both of these files are located in the Documentation director, under the performance files (there are 3 of them).
\\ \\
Naive Bayes:\\
\begin{tabular}{ | l | l | l |}
  \hline
  \textbf{Data Set} & \textbf{Apparent Error} & \textbf{True Error} \\ \hline
  \textit{Mushrooms} & & 91\% \\ \hline
  \textit{Traffic} & & 80\% \\ \hline
\end{tabular}
\\ \\ \\
K-Nearest Neighbors:\\
\begin{tabular}{ | l | l | l | l | l | }
  \hline			
  \textbf{Data Set} & Threshold & k & \textbf{Apparent Error} & \textbf{True Error} \\ \hline
  \textit{Mushrooms} & 1 & 1 & & 74\% \\
  & & 5 & & 88\% \\
  & .1 & 1 & & 98\% \\
  & & 5 & & 94.8\% \\
  & .01 & 1 & & 96\% \\
  & & 5 & & 75\% \\
  & .001 & 1 & & 98\% \\
  & & 5 & & 96.5\% \\ \hline
  \textit{Vehicles} & 1 & 1 & & 0\% \\
  & & 5 & & 93.3\% \\
  & .1 & 1 & & 91.9\% \\
  & & 5 & & 93.5\% \\
  & .01 & 1 & & 89.9\% \\
  & & 5 & & 93.3\% \\
  & .001 & 1 & & 92.8\% \\
  & & 5 & & 93.5\% \\ \hline
\end{tabular}

Since the algorithm error rate performance is given above, I'll only discuss execution performance in the upcoming sections.

\subsection{Naive Bayes}
In the naive bayes approach, I take the basic idea found from the Orange Data Mining Naive Bayes Tutorial page, located here: http://www.ailab.si/orange/doc/ofb/c\_nb.htm in fact, if you look at the code I wrote, it's structured very similar to how they did it here.  My intention was to refactor the code a bit later, for more performance, but the performance is quite good already.\\
\\
To discuss a bit about the design of this algorithm - as I already mentioned it's taken much from the site above.  We count the classes to get the class probability, then we count the attributes to get the attribute class probability.  The general idea is that we structure the ending structure by attribute, then attribute value, then class value with the probability of that particular attribute variable leading to that class value.  Much of the code is really well documented in this fashion, so you can easily see what the ending data structure tends to be.  The equation used is, in Python from the page above:
\begin{lstlisting}[frame=single,language=python]
for i in range(len(domain.attributes)):
  for j in range(len(domain.attributes[i].values)): 
    for k in range(len(domain.classVar.values)):
      pc[i][j][k] = (pc[i][j][k] + self.m * p_class[k])/ \
                    (n_class[k] + self.m)
\end{lstlisting}
For the actual testing, we loop through each attribute in the example, and find the probability of that attribute label with that each class happening - multiply that through each of the attributesin that example and we get values for each class - each class value has a certain probability of happening.  We order this list so that the class label with the greatest probability is at the front and that's the one that example is classified as.  The reason why I keep the other class labels and their probabilities along for the ride is for debugging output or if we wanted to see how likely each class label actually was then we have that option.  My goal in how I wrote this was to return as much information as possible to customize how to output the information.
\subsubsection{Mushroom Data Performance}
\textbf{Execution performance} of this algorithm is incredibly quick, around 0.8 seconds for both training and classifying for all 10 iterations.  While this is a lot more than than the weka version, the performance is quite good enough.\\
\subsubsection{Traffic Data Performance}
\textbf{Execution performance} of this algorithm is incredibly quick, around 0.8 seconds for both training and classifying for all 10 iterations.  While this is a lot more than than the weka version, the performance is quite good enough.\\
\subsection{K-Nearest Neighbors}
I spent a considerable amount of time with this particular algorithm, because of just pure interest in trying to optimize it at least somewhat.  KNN can be a very slow algorithm by default, in that we need to compute the distance from our testing example to the entire trained set.  I did some research about optimizations possible with this - but really settled on an option that I'm fairly happy with.  I introduced in my algorithm a threshold value, where if something is in that general distance that it is ``good enough'' to be closest.  If the threshold is set too high, then my accuracy bombs in this algorithm - but if the threshold is set to around .1 or lower, it seems to be acceptable in terms of performance vs accuracy.  This can further be enhanced by playing with the k-value, which I found that the lower the k as well as the threshold being higher equates out to much better execution performance.  Accuracy is hard, though, so in which looking above we can see that k = 5 gives a little better accuracy than k = 1.  Weka goes with k=1 for default, and that with the treshold makes this a very usable algorithm\\
\\
To discuss a bit about the design of this algorithm - I found an interesting site that that discussed the way to deal with the distance of categorical variables is to associate them with the class probability distribution differences between the two attributes.  These attributes are grouped into a vector of positive distances.  After that is done, we loop through this vector summing over all the attribute distances squared.  The algorithm used is from this page: http://horicky.blogspot.com/2009/05/machine-learning-nearest-neighbor.html and basically equates out to:
\begin{lstlisting}[frame=single,language=python]
sum_over_k(P(class[k] | xai) - P(class[k] | xbi))
\end{lstlisting}
The actual training is pretty much trivial - in fact, really the only training that happens is we call the naive bayes attribute-class-probability, which is used in the distance generation.  So performance of training is pretty much as trivial as it gets - but for testing, it's much harder.\\
\\
For testing, the general algorithm is simple. For a single test row, I compute the distance from that particular element in our space with every training point available in that space.  Once that's done, I sort the resulting list of distances (more specifically it's a (class . distance) a-list), and chop off the first k elements and just return those.  With the threshold value option, the distances between the two points in the graph is compared to our threshold value.  If it's between 0 and that threshold value, I store it in a special variable.  If that list gets up to k faster than going through every node in our training set, I short circuit out of the loops just returning that list of k values.  This is one of many options for trying to limit the computational intensity of this algorithm.  
\subsubsection{KNN Performance}
The general performance of KNN depends on two values in the way I implemented it.  K and our threshold have very large impacts in how this algorithm works.  With k = 5, and the threshold very small (.001), doing 10-fold cross validation could take near an hour to compute.  As I was playing with the options a bit more, I found that the threshold being around .1, and k being 1, performance was quite quick - with fairly decent accuracy to boot.  


\end{document}
